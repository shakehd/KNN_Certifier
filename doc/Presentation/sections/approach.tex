\section{Methodology}
%----------------------------------------------------------------------------------------
\begin{frame}[fragile]{Robustness Certification: Basic idea}
    \begin{itemize}
        \item The possible perturbations of $x$ create a small $\ell_{\infty}$ ball centered in $x$ with radius $\epsilon > 0$:
            \[
                P^{\epsilon}(x) = \{x' \in \mathbb{R}^n\ |\ ||x'- x ||_{\infty} \leq \epsilon \}
            \]x
        \item $P^{\epsilon}(x)$ is called the \emph{Perturbation (or Adversarial)} region of $x$
    \end{itemize}
    \begin{figure}[H]
        \centering
        \usetikzlibrary{arrows.meta, intersections}
        \begin{tikzpicture}[
            scale=0.7,
            dot/.style = {circle, draw, fill=#1, inner sep=3pt, node contents={}}
        ]
            \begin{axis}[
                xmin = 0, xmax = 6,
                ymin = 0, ymax = 6,
                xtick distance = 1,
                ytick distance = 1,
                grid = both,
                minor tick num = 5,
                major grid style = {lightgray},
                minor grid style = {lightgray!25},
                axis x line=center,
                axis y line=center,
                xlabel = {$x$},
                ylabel = {$y$},
                xlabel style={above left},
                ylabel style={below right}
            ]
            \node (m) at (3.5cm,3cm) [dot=gray];
            \draw[fill=gray!70, draw=gray!70,semitransparent] (3cm,2.5cm) rectangle (4cm,3.5cm);
            \node (l_x) at (3.5cm,3cm) {\textbf{\tiny x}};
    		\node (p1) at (2.5cm,5cm) [dot=green];
			\node (p2) at (4cm,4cm) [dot=green];
			\node (p3) at (5cm,1.5cm) [dot=green];
			\node (p4) at (1.5cm,4cm) [dot=green];
			\node (p5) at (1cm,2.5cm) [dot=blue!50];
			\node (p6) at (4cm,5.5cm) [dot=blue!50];
			\node (p6) at (5.5cm,4.5cm) [dot=blue!50];
			\node (p7) at (4cm,0.5cm) [dot=red];
			\node (p8) at (4.5cm,2.5cm) [dot=red];
			\node (p9) at (3cm,2cm) [dot=red];
			\node (p10) at (2.5cm,1cm) [dot=red];
            \end{axis}
        \end{tikzpicture}
    \end{figure}
\end{frame}

\begin{frame}[fragile]{Robustness Certification: Stability}
    \begin{itemize}
        \item Check if exists $x_0 \in P^{\epsilon}(x)$ such that $k\text{-NN}(x) \neq k\text{-NN}(x_0)$
        \item $x_0$ exists $\Rightarrow $ $k$-NN is \textbf{not stable}  on $x$
        \item $x_0$ does not exist $\Rightarrow$ $k$-NN is \textbf{stable}  on $x$
    \end{itemize}
    \begin{figure}[H]
        \centering
        \usetikzlibrary{arrows.meta, intersections}
        \begin{tikzpicture}[
            scale=0.75,
            dot/.style = {circle, draw, fill=#1, inner sep=3pt, node contents={}}
        ]
            \begin{axis}[
                xmin = 0, xmax = 6,
                ymin = 0, ymax = 6,
                xtick distance = 1,
                ytick distance = 1,
                grid = both,
                minor tick num = 5,
                major grid style = {lightgray},
                minor grid style = {lightgray!25},
                axis x line=center,
                axis y line=center,
                xlabel = {$x$},
                ylabel = {$y$},
                xlabel style={above left},
                ylabel style={below right}
            ]
            \node (m) at (3.5cm,3cm) [dot=gray];
            \draw[fill=gray!70, draw=gray!70,semitransparent] (3cm,2.5cm) rectangle (4cm,3.5cm);
            \node (l_x) at (3.5cm,3cm) {\textbf{\tiny x}};

            \node (x_0) at (3.7cm,3.3cm) [black] {$\bullet$};
            \node [below right = -0.32cm and -0.32cm of x_0]{ $\scriptscriptstyle x_0$};
    		\node (p1) at (2.5cm,5cm) [dot=green];
			\node (p2) at (4cm,4cm) [dot=green];
			\node (p3) at (5cm,1.5cm) [dot=green];
			\node (p4) at (1.5cm,4cm) [dot=green];
			\node (p5) at (1cm,2.5cm) [dot=blue!50];
			\node (p6) at (4cm,5.5cm) [dot=blue!50];
			\node (p6) at (5.5cm,4.5cm) [dot=blue!50];
			\node (p7) at (4cm,0.5cm) [dot=red];
			\node (p8) at (4.5cm,2.5cm) [dot=red];
			\node (p9) at (3cm,2cm) [dot=red];
			\node (p10) at (2.5cm,1cm) [dot=red];
            \end{axis}
        \end{tikzpicture}
    \end{figure}
\end{frame}

\begin{frame}{Algorithm overview}
    Given an input samples $x$:
    \begin{enumerate}
        \item<2-> Build a directed graph $G$ where
            \begin{itemize}
                \item Nodes are samples in the training set;
                \item Edges model the relation of \textbf{\emph{being-closer}} to the adversarial region of $x$;
            \end{itemize}
        \item<3-> For each label $\ell$
            \begin{itemize}
                \item Traverse the graph $G$
                \item Find a \textbf{\emph{valid}} path with $k$ samples where $\ell$ is dominant label;
            \end{itemize}
         \item<4-> If more than one dominant label is found
            \begin{itemize}
                \item $k$-NN is not stable on $x$;
                \item otherwise $k$-NN is stable on $x$;
            \end{itemize}
    \end{enumerate}
\end{frame}

\begin{frame}[fragile]{Graph Construction}
    \begin{figure}[h]
        \centering
        \begin{subfigure}{0.5\linewidth}
            \begin{tikzpicture}[>=latex, scale=0.9]
                \centering
                \begin{axis}[
                    xmin = 0, xmax = 2,
                    ymin = 0, ymax = 2,
                    xtick distance = 1,
                    ytick distance = 1,
                    grid = both,
                    minor tick num = 10,
                    major grid style = {lightgray},
                    minor grid style = {lightgray!75},
                    width = 8cm,
                    height = 8cm,
                    axis x line=center,
                    axis y line=center,
                    xlabel = {$x$},
                    ylabel = {$y$},
                    xlabel style={above left},
                    ylabel style={below right}]

                        % draws point
                        \node [green, mark=*](x_2) at (1cm,1.8cm) {$\bullet$};
                        \node [below right = -0.32cm and -0.32cm of x_2]{ $\scriptscriptstyle \sample[2]$};
                        \node [green](x_1) at (2cm,1.6cm) {$\bullet$};
                        \node [below right = -0.32cm and -0.32cm of x_1]{$\scriptscriptstyle \sample[1]$};
                        \node [blue](x_3) at (3cm,1.8cm) {$\bullet$};
                        \node [below right = -0.25cm and -0.50cm of x_3]{$\scriptscriptstyle \sample[3]$};
                        \node [blue](x_4) at (3.2cm,1.8cm) {$\bullet$};
                        \node [below right = -0.60cm and -0.32cm of x_4]{$\scriptscriptstyle \sample[4]$};
                        \node [blue](x_5) at (3cm,3.4cm) {$\bullet$};
                        \node [below right = -0.32cm and -0.32cm of x_5]{$\scriptscriptstyle \sample[5]$};

                        \draw [dashed, visible on=<3-5>] (2cm, 0cm) -- (2cm, 10cm);
                        \draw [dashed, visible on=<6-7>] (3.1cm, 0cm) -- (3.1cm, 10cm);
                        \draw [dashed, visible on=<8-9>] (2.1cm, 0cm) -- (2.1cm, 10cm);

                        \draw[fill=gray!50, draw=gray!50,semitransparent] (1.75cm,1.75cm) rectangle (2.25cm,2.25cm);
                        \node [red](x) at (2cm,2cm) {$\bullet$};
                        \node [below right = -0.32cm and -0.32cm of x]{$\scriptscriptstyle \vinput$};

                        \end{axis}
            \end{tikzpicture}
        \end{subfigure}
        \begin{subfigure}{0.4\linewidth}
            \centering
            \begin{tikzpicture}[>=latex, scale=0.7, baseline={(2.7cm,1cm)}]

                \tikzset{% This is the style settings for nodes
                vertex/.style={circle,minimum size=1cm,fill=#1,draw,
                                        general shadow={fill=gray!60,shadow xshift=1pt,shadow yshift=-1pt}}}


                \node [vertex=green!25, visible on=<2->](s_1) at (3,10) {$\sample[1]$};
                \node [vertex=green!25, visible on=<2->](s_2) at (1,8) {$\sample[2]$};
                \node [vertex=blue!25, visible on=<2->](s_3) at (5,8) {$\sample[3]$};
                \node [vertex=blue!25, visible on=<2->](s_4) at (1,5) {$\sample[4]$};
                \node [vertex=blue!25, visible on=<2->](s_5) at (5,5) {$\sample[5]$};

                \draw [->,very thick, visible on=<10->] (s_1) -- (s_2);
                \draw [->,very thick, visible on=<10->] (s_1) -- (s_3);
                \draw [->,very thick, visible on=<10->] (s_1) to[bend right=70] (s_4);
                \draw [->,very thick, visible on=<10->] (s_1) to[bend left=70] (s_5);

                \draw [->,very thick, visible on=<4->] (s_2) to[bend right=10] (s_3);
                \draw [->,very thick, visible on=<9->] (s_2) to[bend right=10] (s_4);
                \draw [->,very thick, visible on=<11->](s_2) -- (s_5);

                \draw [->,very thick, visible on=<5->] (s_3) to[bend right=10] (s_2);
                \draw [->,very thick, visible on=<7->] (s_3) -- (s_4);
                \draw [->,very thick, visible on=<7->] (s_3) -- (s_4);
                \draw [->,very thick, visible on=<11->](s_3) -- (s_5);

                \draw [->,very thick, visible on=<9->] (s_4) to[bend right=10] (s_2);
                \draw [->,very thick, visible on=<11->] (s_4) -- (s_5);


            \end{tikzpicture}
        \end{subfigure}
    \end{figure}
\end{frame}


\begin{frame}<1-5>[fragile, label=graph-traversal]{Graph Traversal}
    \only<2-3>{$k=1$}
    \only<3>{--- Labels = $\{\textit{green}\}$}
    \only<3>{--- Stable = YES}
    \only<4-10>{$k=3$}
    \only<6-9>{--- Labels = $\{\textit{green}\}$}
    \only<10>{--- Labels = $\{\textit{green, blue}\}$}
    \only<10>{--- Stable = NO}
    % \only<11->{$k=5$}
    % \only<13>{--- Labels = $\{\textit{blue}\}$}
    % \only<13>{--- Stable = YES}
    \begin{figure}[h]
        \centering
        \begin{subfigure}{0.45\linewidth}
            \begin{tikzpicture}[>=latex, scale=0.7, baseline={(-1cm,-1cm)}]

                \tikzset{% This is the style settings for nodes
                    yshift=-1.5cm,
                    xshift=-1.5cm,
                    vertex/.style={circle,minimum size=0.5cm,fill=#1,draw,
                                        general shadow={fill=gray!60,shadow xshift=1pt,shadow yshift=-1pt}}}


                    \node [vertex=green!25](s_1) at (1,8) {$\sample[1]$};
                    \node [vertex=green!25](s_2) at (-1,6) {$\sample[2]$};
                    \node [vertex=blue!25](s_3) at (3,6) {$\sample[3]$};
                    \node [vertex=blue!25](s_4) at (-1,3) {$\sample[4]$};
                    \node [vertex=blue!25](s_5) at (3,3) {$\sample[5]$};

                    \draw [->,very thick] (s_1) -- (s_2);
                    \draw [->,very thick] (s_1) -- (s_3);
                    \draw [->,very thick] (s_1) to[bend right=70] (s_4);
                    \draw [->,very thick] (s_1) to[bend left=70] (s_5);

                    \draw [->,very thick] (s_2) to[bend right=10] (s_3);
                    \draw [->,very thick] (s_2) to[bend right=10] (s_4);
                    \draw [->,very thick](s_2) -- (s_5);

                    \draw [->,very thick] (s_3) to[bend right=10] (s_2);
                    \draw [->,very thick] (s_3) -- (s_4);
                    \draw [->,very thick] (s_3) -- (s_4);
                    \draw [->,very thick](s_3) -- (s_5);

                    \draw [->,very thick] (s_4) to[bend right=10] (s_2);
                    \draw [->,very thick] (s_4) -- (s_5);

            \end{tikzpicture}
        \end{subfigure}
        \onslide<2->{\rulesep}
        \begin{subfigure}{0.4\linewidth}
            \centering
            \begin{tikzpicture}[>=latex, scale=0.7, baseline={(2.7cm,1cm)}]

                \tikzset{% This is the style settings for nodes
                vertex/.style={circle,minimum size=0.5cm,fill=#1,draw,
                                        general shadow={fill=gray!60,shadow xshift=1pt,shadow yshift=-1pt}},
                cross/.style={
                    decoration={markings, mark=at position 0.5 with {
                        \node[scale=1.5, inner sep=0pt] {$\times$};
                    }},
                    postaction={decorate}
                }}

                \node [vertex=green!25, visible on=<2->](s_1) at (3,10) {$\sample[1]$};
                \node [vertex=green!25, visible on=<5-10>](s_12) at (1,8.5) {$\sample[2]$};
                \node [vertex=blue!25, visible on=<5-10>](s_123) at (1,6.5) {$\sample[3]$};
                \node [vertex=blue!25, visible on=<7-8>](s_124) at (3,6.5) {$\sample[4]$};
                \node [vertex=blue!25, visible on=<9-10>](s_13) at (3,8) {$\sample[3]$};
                \node [vertex=blue!25, visible on=<9-10>](s_134) at (3,6) {$\sample[4]$};
                % \node [vertex=green!25, visible on=<12->](s_12_) at (3,8) {$\sample[2]$};
                % \node [vertex=blue!25, visible on=<12->](s_123_) at (3,6) {$\sample[3]$};
                % \node [vertex=blue!25, visible on=<12->](s_1234_) at (3,4) {$\sample[4]$};
                % \node [vertex=blue!25, visible on=<12->](s_12345_) at (3,2) {$\sample[5]$};

                \draw [->,very thick, visible on=<5-10>] (s_1) -- (s_12);
                \draw [->,very thick, visible on=<5-10>] (s_12) -- (s_123);
                \draw [->,very thick, visible on=<7>] (s_12) -- (s_124);
                \draw [cross, ->,very thick, visible on=<8>] (s_12) -- (s_124);
                \draw [->,very thick, visible on=<9-10>] (s_1) -- (s_13);
                \draw [->,very thick, visible on=<9-10>] (s_13) -- (s_134);
                % \draw [->,very thick, visible on=<12->] (s_1) -- (s_12_);
                % \draw [->,very thick, visible on=<12->] (s_12_) -- (s_123_);
                % \draw [->,very thick, visible on=<12->] (s_123_) -- (s_1234_);
                % \draw [->,very thick, visible on=<12->] (s_1234_) -- (s_12345_);

            \end{tikzpicture}
        \end{subfigure}
    \end{figure}
\end{frame}

\begin{frame}<1>[fragile, label=validity-region]{Graph Construction}
    \begin{figure}[h]
        \centering
        \begin{tikzpicture}[>=latex, scale=0.9]
            \centering
            \begin{axis}[
                xmin = 0, xmax = 2,
                ymin = 0, ymax = 2,
                xtick distance = 1,
                ytick distance = 1,
                grid = both,
                minor tick num = 10,
                major grid style = {lightgray},
                minor grid style = {lightgray!75},
                width = 8cm,
                height = 8cm,
                axis x line=center,
                axis y line=center,
                xlabel = {$x$},
                ylabel = {$y$},
                xlabel style={above left},
                ylabel style={below right}]

                    % draws point
                    \node [green, mark=*](x_2) at (1cm,1.8cm) {$\bullet$};
                    \node [below right = -0.32cm and -0.32cm of x_2]{ $\scriptscriptstyle \sample[2]$};
                    \node [green](x_1) at (2cm,1.6cm) {$\bullet$};
                    \node [below right = -0.32cm and -0.32cm of x_1]{$\scriptscriptstyle \sample[1]$};
                    \node [blue](x_3) at (3cm,1.8cm) {$\bullet$};
                    \node [below right = -0.25cm and -0.50cm of x_3]{$\scriptscriptstyle \sample[3]$};
                    \node [blue](x_4) at (3.2cm,1.8cm) {$\bullet$};
                    \node [below right = -0.60cm and -0.32cm of x_4]{$\scriptscriptstyle \sample[4]$};
                    \node [blue](x_5) at (3cm,3.4cm) {$\bullet$};
                    \node [below right = -0.32cm and -0.32cm of x_5]{$\scriptscriptstyle \sample[5]$};

                    \draw[fill=green, draw=green!50,semitransparent, visible on=<1>] (1.75cm,1.75cm) rectangle (2.1cm,2.25cm);
                    \draw[fill=blue, draw=blue!50,semitransparent, visible on=<2>] (2.1cm,1.75cm) rectangle (2.25cm,2.25cm);
                    % \draw[fill=blue, draw=blue!50,semitransparent, visible on=<3>] (1.75cm,1.75cm) rectangle (2.1cm,2.25cm);


                    \draw[fill=gray!50, draw=gray!50,semitransparent] (1.75cm,1.75cm) rectangle (2.25cm,2.25cm);
                    \node [red](x) at (2cm,2cm) {$\bullet$};
                    \node [below right = -0.32cm and -0.32cm of x]{$\scriptscriptstyle \vinput$};

                    \end{axis}
        \end{tikzpicture}
    \end{figure}
\end{frame}

\againframe<6-9>{graph-traversal}
\againframe<2>{validity-region}
\againframe<10>{graph-traversal}
% \againframe<10-12>{graph-traversal}
% \againframe<3>{validity-region}
% \againframe<13->{graph-traversal}


\begin{frame}{Graph Traversal-Summary}

    \begin{itemize}
        \item Start traversal of the graph from sample $\sample[i]$ not dominated by any other samples;

        \item Consider only the paths $[\sample[1], \sample[2], \ldots, \sample[k]]$ such that
            $\sample[1], \sample[2], \ldots, \sample[k]$ are the closest samples to some $x' \in P^{\epsilon}(\vinput)$ than any other samples;
    \end{itemize}
\end{frame}


